\documentclass[reqno]{article}
\usepackage{../format_doc}

\begin{document}
\title{Hexagonal phase field lattice}
\author{Lucas Myers}
\maketitle

\section{Hexagonal lattice phase-field and free energy}
To begin, we create a scalar phase field $\psi$, which roughly corresponds to a time-average of the mass-density of atoms in the lattice.
Hence, we would like to write down a form which has mass concentrated in a hexagonal lattice pattern.
This looks like:
\begin{equation} \label{eq:hexagonal-lattice-configuration}
    \psi(\mathbf{r})
    =
    \psi_0
    +
    \sum_n A_0 e^{i \, \mathbf{q}_n \cdot \mathbf{r}}
\end{equation}
with $\psi_0$ some constant, $A_0$ an amplitude, and each $\mathbf{q}_n$ a lattice vector, given by:
\begin{equation}
    \mathbf{q}_1 = \hat{\mathbf{y}}, \hspace{0.5cm}
    \mathbf{q}_2 = \tfrac{\sqrt{3}}{2} \hat{\mathbf{x}} - \tfrac12 \hat{\mathbf{y}}, \hspace{0.5cm}
    \mathbf{q}_3 = -\tfrac{\sqrt{3}}{2} \hat{\mathbf{x}} - \tfrac12 \hat{\mathbf{y}}
\end{equation}
We will use this as an initial configuration to make sure that we can create a stable configuration and iterate it in time.

The free energy corresponding to a hexagonal lattice is given by:
\begin{equation}
    \mathcal{F}[\psi]
    =
    \int_\Omega \left[
    \frac12 \left[ \left(\nabla^2 + 1\right) \psi \right]^2
    + \frac{\epsilon}{2} \psi^2
    + \frac14 \psi^4
    \right] dV
\end{equation}
where, if $\epsilon > 0$ the only stable solution is $\psi = \psi_0$, whereas $\epsilon < 0$ admits periodic solutions which we are interested in.
Given this free energy, the corresponding time evolution is given by:
\begin{equation}
    \frac{\partial \psi}{\partial t}
    =
    \nabla^2 \frac{\delta F}{\delta \psi}
\end{equation}
Writing this out explicitly gives:
\begin{equation}
\begin{split}
    \frac{\partial \psi}{\partial t}
    &=
    \nabla^2 \left[\left(\nabla^2 + 1\right)^2 \psi
    + \epsilon \psi
    + \psi^3
    \right] \\
    &=
    \nabla^2 \left(\nabla^2 + 1 \right)^2 \psi
    + \epsilon \nabla^2 \psi
    + \nabla^2 \psi^3
\end{split}
\end{equation}

\section{Finite element method}
To begin, we expand the equation as much as possible:
\begin{equation}
    \begin{split}
        \frac{\partial \psi}{\partial t}
        &=
        \nabla^6 \psi
        + 2 \nabla^4 \psi
        + \left(1 + \epsilon \right) \nabla^2 \psi
        + \nabla^2 \psi^3 \\
        &=
        \nabla^6 \psi
        + 2 \nabla^4 \psi
        + \left(1 + \epsilon \right) \nabla^2 \psi
        + 3 \psi^2 \nabla^2 \psi
        + 6 \psi \left( \nabla \psi \right)^2
    \end{split}
\end{equation}
For technical reasons associated with the use of piecewise polynomials for the finite element method, we choose to define the following auxiliary variables:
\begin{equation}
    \chi \coloneqq \nabla^2 \psi, \hspace{0.3cm}
    \phi \coloneqq \nabla^2 \chi = \nabla^4 \psi
\end{equation}
Substituting these auxiliary variables into our original equations yields:
\begin{equation} \label{eq:expanded-phase-field}
    \begin{split}
        \frac{\partial \psi}{\partial t}
        =
        \nabla^2 \phi
        + 2 \phi
        + \left(1 + \epsilon \right) \chi
        + 3 \psi^2 \chi
        + 6 \psi \left(\nabla \psi \right)^2
    \end{split}
\end{equation}
Then we have three coupled second-order equations.

\subsection{Time discretization}
We may solve this equation numerically using the finite element method.
To begin, we discretize in time.
For this, we use a semi-implicit method parameterized by $\theta$ as in \href{https://www.dealii.org/current/doxygen/deal.II/step_23.html}{step 23 of the deal.II tutorials}.
This gives:
\begin{equation} \label{time-discretization}
    \frac{\psi_n - \psi_{n - 1}}{\delta t}
    =
    \begin{multlined}[t]
    \theta \left[
        \nabla^2 \phi_n
        + 2 \phi_n
        + \left(1 + \epsilon + 3 \psi_n^2 \right) \chi_n
        + 6 \psi_n \left(\nabla \psi_n \right)^2
    \right] \\
    + \left(1 - \theta \right) \left[
        \nabla^2 \phi_{n - 1}
        + 2 \phi_{n - 1}
        + \left(1 + \epsilon + 3 \psi_{n - 1}^2 \right) \chi_{n - 1}
        + 6 \psi_n \left(\nabla \psi_{n - 1} \right)^2
    \right]
    \end{multlined}
\end{equation}
Here $\delta t$ is the discrete timestep and $\theta$ is the discretization parameter: $\theta = 1$ corresponds to a fully-implicit method, $\theta = 0$ is a fully explicit method, and $\theta = 1/2$ is a Crank-Nicolson method.
Of course, the definitions of the auxiliary variables still hold for timestep $n$ and $n - 1$.

\subsection{Linearizing the equation}
Given that the equation is nonlinear, we must use Newton-Rhapson method to solve for each timestep.
Because we have a series of three coupled equations, the residual is a three-component function defined at every point.
The components are then given by:
\begin{equation}
    \begin{split}
        R_1(\psi_n, \chi_n, \phi_n)
        &=
        \begin{multlined}[t]
        \psi_n
        - \delta t \, \theta \left[
            \nabla^2 \phi_n
            + 2 \phi_n
            + \left(1 + \epsilon + 3 \psi_n^2 \right) \chi_n
            + 6 \psi_n \left(\nabla \psi_n \right)^2
        \right] \\
        - \psi_{n - 1} 
        - \delta t \, \left(1 - \theta \right) \left[
            \nabla^2 \phi_{n - 1}
            + 2 \phi_{n - 1}
            + \left(1 + \epsilon + 3 \psi_{n - 1}^2 \right) \chi_{n - 1}
            + 6 \psi_{n - 1} \left(\nabla \psi_{n - 1} \right)^2
        \right]
        \end{multlined}\\
        R_2(\psi_n, \chi_n, \phi_n)
        &=
        \chi_n - \nabla^2 \psi_n \\
        R_3(\psi_n, \chi_n, \phi_n)
        &=
        \phi_n - \nabla^2 \chi_n
    \end{split}
\end{equation}
Since our residual is vector-valued, the Gateaux derivative will be a matrix which acts on a vector-valued deviation $\left[ \delta \psi, \delta \chi, \delta \phi \right]^T$.
Taking the Gateaux derivative one row at a time yields:
\begin{equation}
    \begin{split}
    dR_1(\psi, \chi, \phi; \; \delta \psi, \delta \chi, \delta \phi)
    &=
    \frac{d}{d \tau} R_1(\psi + \tau \delta \psi, \chi + \tau \delta \chi, \phi + \tau \delta \phi) \biggr|_{\tau = 0} \\
    &=
    \frac{d}{d\tau}
    \left[
    \psi_n + \tau \delta \psi_n
    - 
    \begin{multlined}[t]
        \delta t \, \theta \bigl[
        \nabla^2 \left(\phi_n + \tau \delta \phi_n \right)
        + 2 \left( \phi_n + \tau \delta \phi_n \right) \\
        + \left(1 + \epsilon + 3 \left(\psi_n + \tau \delta \psi_n\right)^2 \right) \left( \chi_n + \tau \delta \chi_n \right) \\
        + 6 \left( \psi_n + \tau \delta \psi_n \right) \left(\nabla \left( \psi_n + \tau \delta \psi_n \right) \right)^2
    \bigr] 
    \end{multlined}
    \right]_{\tau = 0} \\
    &=
    \delta \psi_n
    - 
    \begin{multlined}[t]
        \delta t \, \theta \bigl[
        \nabla^2 \delta \phi_n
        + 2 \delta \phi_n \\
        + \left(1 + \epsilon + 3 \psi_n^2 \right) \delta \chi_n
        + 6 \psi_n \chi_n \, \delta \psi_n \\
        + 6 \left(\nabla \psi_n\right)^2 \delta \psi_n
        + 12 \psi_n \left(\nabla \psi_n \right) \cdot \nabla \delta \psi_n
    \bigr]
    \end{multlined} \\
    dR_2(\psi, \chi, \phi; \; \delta \psi, \delta \chi, \delta \phi)
    &=
    \delta \chi_n - \nabla^2 \delta \psi_n \\
    dR_3(\psi, \chi, \phi; \; \delta \psi, \delta \chi, \delta \phi)
    &=
    \delta \phi_n - \nabla^2 \delta \chi_n
    \end{split}
\end{equation}
Given this derivative of the residual, then \href{https://www.wikiwand.com/en/Newton's_method#In_a_Banach_space}{Newton-Rhapson method in functional space} reads:
\begin{equation}
    \begin{split}
        dR(\Psi_n) \delta \Psi_n
        &= -R(\Psi_n) \\
        \Psi_{n + 1} 
        &= \Psi_n + \delta \Psi_n
    \end{split}
\end{equation}
where we have defined:
\begin{equation}
    \Psi
    =
    \begin{bmatrix}
        \psi \\
        \chi \\
        \phi
    \end{bmatrix}
\end{equation}
$R$ is the vector residual, $\delta \Psi$ is the vector variation of $\Psi$ and:
\begin{equation}
    dR(\Psi)
    =
    \begin{bmatrix}
        1 - \delta t \theta \left[6 \psi \chi + 6 \left(\nabla \psi\right)^2 + 12 \psi \left(\nabla \psi\right) \cdot \nabla \right]
        &-\delta t \theta \left(1 + \epsilon + 3 \psi \right)
        &-\delta t \theta \left(\nabla^2 + 2\right) \\
        -\nabla^2 &1 &0 \\
        0 &-\nabla^2 &1
    \end{bmatrix}
\end{equation}

\subsection{Space discretization}
For this, we introduce a vector test function $A$:
\begin{equation}
    A
    =
    \begin{bmatrix}
        \alpha \\
        \beta \\
        \gamma
    \end{bmatrix}
\end{equation}
Then our linear equation looks like:
\begin{equation}
    A^T \, dR \, \delta \Psi
    =
    -A^T \, R
\end{equation}
The left-hand side is given by:
\begin{equation}
    \begin{split}
        A^T \, dR \, \delta \Psi
        &=
        \begin{multlined}[t]
            \left<\alpha, \delta \psi_n \right>
            - \delta t \theta \bigl[ \left< \alpha, \nabla^2 \delta \phi \right>
                + 2 \left< \alpha, \delta \phi_n \right> \\
                + \left(1 + \epsilon\right) \left< \alpha, \delta \chi_n \right>
                + 3 \left< \alpha, \psi_n^2 \delta \chi_n \right> 
                + 6 \left< \alpha, \psi_n \, \chi_n \, \delta \psi_n \right> \\
                + 6 \left<\alpha, \left(\nabla \psi_n\right)^2 \delta \psi_n \right>
                + 12 \left< \alpha, \psi_n \left(\nabla \psi_n \right) \cdot \nabla \delta \psi_n \right>
            \bigr] \\
            + \left< \beta, \delta \chi_n \right>
            - \left< \beta, \nabla^2 \delta \psi_n \right> \\
            + \left< \gamma, \delta \phi_n \right>
            - \left< \gamma, \nabla^2 \delta \chi_n \right>
        \end{multlined} \\
        &=
        \begin{multlined}[t]
            \left<\alpha, \delta \psi_n \right>
            - \delta t \theta \bigl[ -\left< \nabla \alpha, \nabla \delta \phi \right>
                + 2 \left< \alpha, \delta \phi_n \right> \\
                + \left(1 + \epsilon\right) \left< \alpha, \delta \chi_n \right>
                + 3 \left< \alpha, \psi_n^2 \delta \chi_n \right> 
                + 6 \left< \alpha, \psi_n \, \chi_n \, \delta \psi_n \right> \\
                + 6 \left<\alpha, \left(\nabla \psi_n\right)^2 \delta \psi_n \right>
                + 12 \left< \alpha, \psi_n \left(\nabla \psi_n \right) \cdot \nabla \delta \psi_n \right>
            \bigr] \\
            + \left< \beta, \delta \chi_n \right>
            + \left< \nabla \beta, \nabla \delta \psi_n \right> \\
            + \left< \gamma, \delta \phi_n \right>
            + \left< \nabla \gamma, \nabla \delta \chi_n \right>
        \end{multlined}
    \end{split}
\end{equation}
and the right-hand side is given by:
\begin{equation}
    \begin{split}
        -A^T R
        &=
        \begin{multlined}[t]
            -\left<\alpha, \psi_n \right>
            + \delta t \theta \bigl[
                \left<\alpha, \nabla^2 \phi_n \right> 
                + 2 \left< \alpha, \phi_n \right>  
                + \left(1 + \epsilon\right) \left< \alpha, \chi_n \right> \\
                + 3 \left< \alpha, \psi_n^2 \chi_n \right>
                + 6 \left< \alpha, \psi_n \left(\nabla \psi_n \right)^2 \right>
                \bigr] \\
            + \left< \alpha, \psi_{n - 1} \right>
            + \delta t \left(1 - \theta \right) \bigl[
                \left<\alpha, \nabla^2 \phi_{n - 1} \right>
                + 2 \left< \alpha, \phi_{n - 1} \right> 
                + \left(1 + \epsilon\right) \left<\alpha, \chi_{n - 1} \right> \\
                + 3 \left< \alpha, \psi_{n - 1}^2 \chi_{n - 1} \right> 
                + 6 \left< \alpha, \psi_{n - 1} \left(\nabla \psi_{n - 1}\right)^2 \right>
            \bigr] \\
            - \left<\beta, \chi_n \right>
            + \left<\beta, \nabla^2 \psi_n\right>
            - \left< \gamma, \phi_n \right>
            + \left< \gamma, \nabla^2 \chi_n \right>
        \end{multlined} \\
        &=
        \begin{multlined}[t]
            -\left<\alpha, \psi_n \right>
            + \delta t \theta \bigl[
                -\left<\nabla\alpha, \nabla \phi_n \right> 
                + 2 \left< \alpha, \phi_n \right>  
                + \left(1 + \epsilon\right) \left< \alpha, \chi_n \right> \\
                + 3 \left< \alpha, \psi_n^2 \chi_n \right>
                + 6 \left< \alpha, \psi_n \left(\nabla \psi_n \right)^2 \right>
                \bigr] \\
            + \left< \alpha, \psi_{n - 1} \right>
            + \delta t \left(1 - \theta \right) \bigl[
                -\left<\nabla \alpha, \nabla \phi_{n - 1} \right>
                + 2 \left< \alpha, \phi_{n - 1} \right> 
                + \left(1 + \epsilon\right) \left<\alpha, \chi_{n - 1} \right> \\
                + 3 \left< \alpha, \psi_{n - 1}^2 \chi_{n - 1} \right> 
                + 6 \left< \alpha, \psi_{n - 1} \left(\nabla \psi_{n - 1}\right)^2 \right>
            \bigr] \\
            - \left<\beta, \chi_n \right>
            - \left<\nabla\beta, \nabla \psi_n\right>
            - \left< \gamma, \phi_n \right>
            - \left< \nabla\gamma, \nabla \chi_n \right>
        \end{multlined}
    \end{split}
\end{equation}
Now our shape functions are vector-valued, composed like:
\begin{equation}
    \eta_i
    =
    \begin{bmatrix}
        \eta_{i, \psi} \\
        \eta_{i, \chi} \\
        \eta_{i, \phi}
    \end{bmatrix}
\end{equation}
Note that, for primitive elements like we will use, only one component will be nonzero for any particular $i$.
In any case, supposing we assume our solution vector can be written as a linear combinations of primitive test functions, and stipulating that the equation hold for any test function, we get the following left-hand side:
\begin{equation}
    A^T \, dR \, \delta \Psi
    =
    \sum_{j}
    \begin{multlined}[t]
    \Biggl[
        \left<\eta_{i, \psi}, \eta_{j, \psi} \right>
        - \delta t \theta \bigl[ -\left< \nabla \eta_{i, \psi}, \nabla \eta_{j, \phi} \right>
            + 2 \left< \eta_{i, \psi}, \eta_{j, \phi} \right> \\
            + \left(1 + \epsilon\right) \left< \eta_{i, \psi}, \eta_{j, \chi} \right>
            + 3 \left< \eta_{i, \psi}, \psi_n^2 \eta_{j, \chi} \right> 
            + 6 \left< \eta_{i, \psi}, \psi_n \, \chi_n \, \eta_{j, \psi} \right> \\
            + 6 \left<\eta_{i, \psi}, \left(\nabla \psi_n\right)^2 \eta_{j, \psi} \right>
            + 12 \left< \eta_{i, \psi}, \psi_n \left(\nabla \psi_n \right) \cdot \nabla \eta_{j, \psi} \right>
        \bigr] \\
        + \left< \eta_{i, \chi}, \eta_{j, \chi} \right>
        + \left< \nabla \eta_{i, \chi}, \nabla \eta_{j, \psi} \right> \\
        + \left< \eta_{i, \phi}, \eta_{j, \phi} \right>
        + \left< \nabla \eta_{i, \phi}, \nabla \eta_{j, \chi} \right>
    \Biggr]
    \delta \Psi_j
    \end{multlined}
\end{equation}
Finally, we may write our right-hand side as:
\begin{equation}
    -A^T R
    =
    \begin{multlined}[t]
        -\left<\eta_{i, \psi}, \psi_n \right>
        + \delta t \theta \bigl[
            -\left<\nabla\eta_{i, \psi}, \nabla \phi_n \right> 
            + 2 \left< \eta_{i, \psi}, \phi_n \right>  
            + \left(1 + \epsilon\right) \left< \eta_{i, \psi}, \chi_n \right> \\
            + 3 \left< \eta_{i, \psi}, \psi_n^2 \chi_n \right>
            + 6 \left< \eta_{i, \psi}, \psi_n \left(\nabla \psi_n \right)^2 \right>
            \bigr] \\
        + \left< \eta_{i, \psi}, \psi_{n - 1} \right>
        + \delta t \left(1 - \theta \right) \bigl[
            -\left<\nabla \eta_{i, \psi}, \nabla \phi_{n - 1} \right>
            + 2 \left< \eta_{i, \psi}, \phi_{n - 1} \right> 
            + \left(1 + \epsilon\right) \left<\eta_{i, \psi}, \chi_{n - 1} \right> \\
            + 3 \left< \eta_{i, \psi}, \psi_{n - 1}^2 \chi_{n - 1} \right> 
            + 6 \left< \eta_{i, \psi}, \psi_{n - 1} \left(\nabla \psi_{n - 1}\right)^2 \right>
        \bigr] \\
        - \left<\eta_{i, \chi}, \chi_n \right>
        - \left<\nabla\eta_{i, \chi}, \nabla \psi_n\right>
        - \left< \eta_{i, \phi}, \phi_n \right>
        - \left< \nabla\eta_{i, \phi}, \nabla \chi_n \right>
    \end{multlined}
\end{equation}
Finally, note that our matrix can be written in block form as:
\begin{equation}
    \begin{bmatrix}
        B &C &D \\
        L_{\psi} &M_{\chi} &0 \\
        0 &L_{\chi} &M_{\phi}
    \end{bmatrix}
    \begin{bmatrix}
        \delta \psi \\
        \delta \chi \\
        \delta \phi
    \end{bmatrix}
    =
    \begin{bmatrix}
        F \\
        G \\
        H
    \end{bmatrix}
\end{equation}
where $F$, $G$, $H$ are the different components of $-A^T R$:
\begin{equation}
    \begin{split}
        F_i
        &=
        \begin{multlined}[t]
            \biggl<\eta_{i, \psi}, 
            -\psi_n + \psi_{n - 1} \\
            + \delta t \biggl[ 
            \theta \left( 2 \phi_n + (1 + \epsilon) \chi_n + 3 \psi_n^2 \chi_n + 6 \psi_n \left(\nabla \psi_n \right)^2
            \right) \\
            + \left(1 - \theta \right) \left( 2 \phi_{n - 1} + (1 + \epsilon) \chi_{n - 1} + 3 \psi_{n - 1}^2 \chi_{n - 1} + 6 \psi_{n - 1} \left(\nabla \psi_{n - 1}\right)^2
            \right)
            \biggr]\biggr> \\
            - \delta t \left< \nabla \eta_{i, \psi}, \theta \nabla \phi_n + \left(1 - \theta\right) \nabla \phi_{n - 1} \right>
        \end{multlined}\\
        G_i
        &=
        - \left<\eta_{i, \chi}, \chi_n \right>
        - \left< \nabla \eta_{i, \chi}, \nabla \psi_n \right> \\
        H_i
        &=
        - \left< \eta_{i, \phi}, \phi_n \right>
        - \left<\nabla \eta_{i, \phi}, \nabla \chi_n \right>
    \end{split}
\end{equation}

$\delta \psi$, $\delta \chi$, $\delta \phi$ are the finite element vectors for each of the variations, and then:
\begin{equation}
    \begin{split}
        B_{ij} 
        &= \left<\eta_{i,\psi}, \eta_{j, \psi}\right> 
        - 6 \delta t \, \theta \left[ \left< \eta_{i, \psi}, \psi_n \, \chi_n \eta_{j, \psi} \right>
        + \left< \eta_{i, \psi}, \left(\nabla \psi_n\right)^2 \eta_{j, \psi} \right>
        + 2 \left< \eta_{i, \psi}, \psi_n \left(\nabla \psi_n \right) \cdot \nabla \eta_{j, \psi} \right> \right] \\
        C_{ij}
        &= -\delta t \, \theta \left[ \left(1 + \epsilon\right) \left<\eta_{i, \psi}, \eta_{j, \chi}\right>
        + 3 \left< \eta_{i, \psi}, \psi_n^2 \eta_{j, \chi} \right>
        \right] \\
        D_{ij}
        &= -\delta t \, \theta \left[ -\left<\nabla \eta_{i, \psi}, \nabla \eta_{j, \phi}\right>
        + 2 \left< \eta_{i, \psi}, \eta_{j, \phi} \right> 
        \right] \\
        L_{ij}
        &= \left< \nabla \eta_{i}, \nabla \eta_{j} \right> \\
        M_{ij}
        &= \left< \eta_i, \eta_j \right>
    \end{split}
\end{equation}
where $L$ is something like the Laplacian operator, and $M$ is the mass matrix.

\subsection{Preconditioning}
For this, we would like to have a $2\times 2$ block so that we may apply the Schur complement method.
Let's write the matrix as:
\begin{equation}
    \begin{bmatrix}
        B &X \\
        Y &A
    \end{bmatrix}
\end{equation}
where:
\begin{equation}
    X
    =
    \begin{bmatrix}
        C &D
    \end{bmatrix}, \:
    Y 
    = 
    \begin{bmatrix}
        L_\psi \\
        0
    \end{bmatrix}, \:
    A
    =
    \begin{bmatrix}
        M_\chi &0 \\
        L_\chi &M_\phi
    \end{bmatrix}
\end{equation}
Call $S = B - X A^{-1} Y$.
Then the inverse of the system matrix is given by:
\begin{equation}
    \begin{bmatrix}
        B &X \\
        Y &A
    \end{bmatrix}^{-1}
    =
    \begin{bmatrix}
        S^{-1} &-S^{-1}X A^{-1} \\
        -A^{-1}Y S^{-1} &A^{-1} + A^{-1}YS^{-1}X A^{-1}
    \end{bmatrix}
\end{equation}
We may write down $S$ more explicitly by considering $A$ and computing its inverse:
\begin{equation}
    A^{-1}
    =
    \begin{bmatrix}
        M_\chi^{-1} &0 \\
        -M_\phi^{-1} L_\chi M_\chi^{-1} &M_\phi^{-1}
    \end{bmatrix}
\end{equation}
This is just done with the Schur complement method and noting that $M_\chi$ is the Schur complement of $A$ with respect to $M_\phi$.
This is getting somewhat complicated, so considering that $C$ and $D$ are scaled by the timestep $\delta t$, let's assume those contributions are small. 
Additionally, $B = M_\psi + \mathcal{O}(\delta t)$ so let's assume $\delta t$ is small as well.
Given this, $S \approx M_\psi$ and then:
\begin{equation}
    R^{-1}
    =
    -A^{-1} Y S^{-1}
    =
    \begin{bmatrix}
        -M_\chi^{-1} L_\psi M_\psi^{-1} \\
        M_\phi^{-1} L_\chi M_\chi^{-1} L_\psi M_\psi^{-1}
    \end{bmatrix}
\end{equation}
Then our preconditioner becomes:
\begin{equation}
    \begin{split}
        P^{-1}
        &=
        \begin{bmatrix}
            M_\psi^{-1} &0 \\
            R^{-1} &A^{-1}
        \end{bmatrix} \\
        &=
        \begin{bmatrix}
            M_\psi^{-1} &0 &0\\
            -M_\chi^{-1} L_\psi M_\psi^{-1} &M_\chi^{-1} &0 \\
            M_\phi^{-1} L_\chi M_\chi^{-1} L_\psi M_\psi^{-1} &-M_\phi^{-1} L_\chi M_\chi^{-1} &M_\phi^{-1}
        \end{bmatrix}
    \end{split}
\end{equation}
The only requirements for this preconditioner are that we multiply by Laplacians and invert mass matrices.
If this does not work, we may come back to it and try to simplify $B$, $C$, and $D$ less.

\subsection{Preconditioning $B$ effectively}
The preconditioner proposed above did not work well.
For 6 refinements, it caused slightly (~10\%) fewer iterations, but for 7 it got about an order of magnitude worse.
Hence, we will try the next most complicated thing which is to still assume $X \approx 0$, but then try to approximate $B^{-1}$.
In this case, $S \approx B$, and $S^{-1} \approx \widetilde{B^{-1}}$:
\begin{equation}
    P^{-1}
    =
    \begin{bmatrix}
        \widetilde{B^{-1}} &0 &0\\
        -M_\chi^{-1} L_\psi \widetilde{B^{-1}} &M_\chi^{-1} &0 \\
        M_\phi^{-1} L_\chi M_\chi^{-1} L_\psi \widetilde{B^{-1}} &-M_\phi^{-1} L_\chi M_\chi^{-1} &M_\phi^{-1}
    \end{bmatrix}
\end{equation}

\subsection{Including $X$}
The above does not seem to work very effectively either. 
Hence, let us just do the most straightforward thing and calculate $S$ explicitly, by first calculating $X A^{-1} Y$:
\begin{equation}
    \begin{split}
        X A^{-1} Y
        &=
        \begin{bmatrix}
            C &D
        \end{bmatrix}
        \begin{bmatrix}
            M_\chi^{-1} &0 \\
            -M_\phi^{-1} L_\chi M_\chi^{-1} &M_\phi^{-1}
        \end{bmatrix}
        \begin{bmatrix}
            L_\psi \\
            0
        \end{bmatrix}\\
        &=
        \begin{bmatrix}
            C &D
        \end{bmatrix}
        \begin{bmatrix}
            M_\chi^{-1} L_\psi \\
            -M_\phi^{-1} L_\chi M_\chi^{-1} L_\psi
        \end{bmatrix} \\
        &=
        C M_\chi^{-1} L_\psi
        - D M_\phi^{-1} L_\chi M_\chi^{-1} L_\psi
    \end{split}
\end{equation}
Hence, we may calculate $S$ directly.
Finally, we may calculate $X A^{-1}$ and $A^{-1} Y$ explicitly:
\begin{equation}
    \begin{split}
        X A^{-1}
        &=
        \begin{bmatrix}
            C &D
        \end{bmatrix}
        \begin{bmatrix}
            M_\chi^{-1} &0 \\
            - M_\phi^{-1} L_\chi M_\chi^{-1} &M_\phi^{-1}
        \end{bmatrix} \\
        &= 
        \begin{bmatrix}
            C M_\chi^{-1} - D M_\phi^{-1} L_\chi M_\chi^{-1} &D M_\phi^{-1}
        \end{bmatrix}
    \end{split}
\end{equation}
\begin{equation}
    \begin{split}
        A^{-1} Y
        &=
        \begin{bmatrix}
            M_\chi^{-1} &0 \\
            - M_\phi^{-1} L_\chi M_\chi^{-1} &M_\phi^{-1}
        \end{bmatrix}
        \begin{bmatrix}
            L_\psi \\
            0
        \end{bmatrix} \\
        &=
        \begin{bmatrix}
            M_\chi^{-1} L_\psi \\
            -M_\phi^{-1} L_\chi M_\chi^{-1} L_\psi
        \end{bmatrix}
    \end{split}
\end{equation}
The last thing to calculate:
\begin{equation}
    \begin{split}
        A^{-1} Y S^{-1} X A^{-1}
        &=
        \begin{bmatrix}
            M_\chi^{-1} L_\psi \\
            -M_\phi^{-1} L_\chi M_\chi^{-1} L_\psi
        \end{bmatrix}
        \begin{bmatrix}
            S^{-1}
        \end{bmatrix}
        \begin{bmatrix}
            C M_\chi^{-1} - D M_\phi^{-1} L_\chi M_\chi^{-1} &D M_\phi^{-1}
        \end{bmatrix} \\
        &=
        \begin{bmatrix}
            M_\chi^{-1} L_\psi S^{-1} \left(C M_\chi^{-1} - D M_\phi^{-1} L_\psi M_\chi^{-1}\right)
            & M_\chi^{-1} L_\psi S^{-1} D M_\phi^{-1} \\
            -M_\phi^{-1} L_\chi M_\chi^{-1} L_\psi S^{-1} \left(C M_\chi^{-1} - D M_\phi^{-1} L_\psi M_\chi^{-1}\right)
            & -M_\phi^{-1} L_\chi M_\chi^{-1} L_\psi S^{-1} D M_\phi^{-1}
        \end{bmatrix}
    \end{split}
\end{equation}
This explicitly gives us the entire inverted matrix.

\section{Simpler Jacobian}
Apparently it is often reasonable to simplify a complicated Jacobian, so long as the Newton iterations still converge to the given residual.
To this end, we define a simplified Jacobian as:
\begin{equation}
    dR(\Psi)
    =
    \begin{bmatrix}
        1 - \delta t \, \theta \left[6 + 12 \, \mathbf{m} \cdot \nabla \right]
        &-\delta t \, \theta \left(4 + \epsilon\right)
        &-\delta t \, \theta \left(\nabla^2 + 2\right) \\
        -\nabla^2 &1 &0 \\
        0 &-\nabla^2 &1
    \end{bmatrix}
\end{equation}
where here $\mathbf{m} = (1, 1, 1)$.
Translated into a weak form, this gives:
\begin{equation}
    \begin{split}
        B_{ij} 
        &= 
        \left<\eta_{i, \psi}, \eta_{i, \psi}\right>
        - 6 \delta t \, \theta \left[
            \left<\eta_{i, \psi}, \eta_{j, \psi}\right>
            + 2 \left<\eta_{i, \psi}, \mathbf{m} \cdot \nabla \eta_{j, \psi}\right>
        \right] \\
        C_{ij}
        &=
        -\delta t \, \theta \left(4 + \eta\right) \left<\eta_{i, \psi}, \eta_{j, \chi}\right> \\
    \end{split}
\end{equation}

\section{Initializing hexagonal lattice}

For this, we need to initialize $\psi$, $\chi$, and $\phi$.
$\psi$ is given in eq. \eqref{eq:hexagonal-lattice-configuration}.
However, $\psi$ is real so that we must only take the real part of the equation to get:
\begin{equation}
    \psi\left(\mathbf{r}\right)
    =
    \psi_0
    + \sum_n A_0 \, \cos \left(\mathbf{q}_n \cdot \mathbf{r} \right)
\end{equation}
The other fields are then given as:
\begin{equation}
    \chi(\mathbf{r})
    =
    \nabla^2 \psi(\mathbf{r})
    =
    - A_0 \sum_n q_n^2 \, \cos \left(\mathbf{q}_n \cdot \mathbf{r} \right)
    = 
    - A_0 \sum_n \cos \left(\mathbf{q}_n \cdot \mathbf{r} \right) 
\end{equation}
and
\begin{equation}
    \phi(\mathbf{r})
    =
    \nabla^2 \chi(\mathbf{r})
    =
    A_0 \sum_n q_n^4 \, \cos \left(\mathbf{q}_n \cdot \mathbf{r} \right) 
    =
    A_0 \sum_n \cos \left(\mathbf{q}_n \cdot \mathbf{r} \right) 
\end{equation}
because each $\mathbf{q}_n$ is a unit vector.

\section{Configurations wtih dislocations}
According to Fei's report, the way to make a dislocation at point $\mathbf{u}_j$ with Burgers vector $\mathbf{b}_j$ is to first define the quantity:
\begin{equation}
    s_{n, j} = \frac{1}{2\pi} \left(\mathbf{q}_n \cdot \mathbf{b}_j\right)
\end{equation}
Given this, we can define a phase distortion given by:
\begin{equation}
    \varphi_{n, j} \left(\mathbf{r}\right)
    =
    s_{n, j} \, \theta_j \left( \mathbf{r} - \mathbf{u}_j \right)
    = s_{n, j} \, \text{atan2}\left(r_y - u_{j, y}, r_x - u_{j, x}\right)
\end{equation}
where $\theta_j$ is the polar angle centered at $\mathbf{u}_j$.
The phase field then becomes:
\begin{equation}
    \psi(\mathbf{r})
    =
    \psi_0
    + A_0 \sum_n e^{i \left(\mathbf{q}_n \cdot \mathbf{r} + \varphi_{n, j}(\mathbf{r})\right)}
\end{equation}
For an arbitrary number of defects, this becomes:
\begin{equation}
    \psi(\mathbf{r})
    =
    \psi_0
    + A_0 \sum_n e^{i \left(\mathbf{q}_n \cdot \mathbf{r} + \sum_j \varphi_{n, j}(\mathbf{r})\right)}
\end{equation}
Given this, we calculate the auxiliary fields:
\begin{equation}
    \begin{split}
        \chi(\mathbf{r})
        &=
        \nabla^2 \psi(\mathbf{r}) \\
        &=
        A_0 \sum_n e^{i \left(\mathbf{q}_n \cdot \mathbf{r} + \sum_j \varphi_{n, j}(\mathbf{r})\right)}
        \left(-q_n^2 + \sum_j \nabla^2 i \varphi_{n, j}\right) \\
        &= -A_0 \sum_n q_n^2 e^{i \left(\mathbf{q}_n \cdot \mathbf{r} + \sum_j \varphi_{n, j}(\mathbf{r})\right)}
    \end{split}
\end{equation}
Here, note that $\varphi_{n, j}$ is proportional to a polar angle centered at $\mathbf{u}_j$.
Additionally, in two dimensions in polar coordinates the Laplacian operator is given by:
\begin{equation}
    \nabla^2
    =
    \frac{\partial^2}{\partial r^2}
    + \frac{1}{r} \frac{\partial}{\partial r}
    + \frac{1}{r^2} \frac{\partial^2}{\partial \theta^2}
\end{equation}
Since this is independent of basis, we can change basis to be centered around $\mathbf{u}_j$ in which case $\nabla^2 \varphi_{n, j} = 0$ is easily calculated.
Similarly:
\begin{equation}
    \phi(\mathbf{r})
    = A_0 \sum_n q_n^4 e^{i \left(\mathbf{q}_n \cdot \mathbf{r} + \sum_j \varphi_{n, j}(\mathbf{r})\right)}
\end{equation}

\section{Calculating configurational stress}
The energy due to an infinitesimal distortion of the phase field is given by:
\begin{equation}
    \mathcal{E}
    =
    \left[
        -\frac{\partial f}{\partial (\partial_{ik} \psi)} \partial_{jk} \psi
        + \left(\partial_k \frac{\partial f}{\partial (\partial_{ik}\psi)}\right) \partial_j \psi
        + \delta_{ij} f
    \right]
    \partial_i u_j
\end{equation}
The configurational stress is the derivative of this with respect to the displacement gradient:
\begin{equation} \label{eq:configurational-stress}
    \begin{split}
        \sigma_{ij}
        &=
        \frac{\partial \mathcal{E}}{\partial (\partial_i u_j)} \\
        &=
        -\frac{\partial f}{\partial (\partial_{ik} \psi)} \partial_{jk} \psi
        + \left(\partial_k \frac{\partial f}{\partial (\partial_{ik}\psi)}\right) \partial_j \psi
        + \delta_{ij} f
    \end{split}
\end{equation}
Which, given the specific free energy density $f$ gives:
\begin{equation}
    \sigma_{ij}
    =
    \left[\partial_i \mathcal{L} \psi \right] \partial_j \psi
    - \left[ \mathcal{L} \psi \right] \partial_{ij} \psi
    + f \delta_{ij}
\end{equation}
where $\mathcal{L} = 1 + \nabla^2$.
Written out in its entirety we get:
\begin{equation}
    \sigma_{ij}
    =
    (\partial_i \psi) (\partial_j \psi)
    + (\partial_i \nabla^2 \psi) (\partial_j \psi)
    - \psi \partial_{ij} \psi
    - (\nabla^2 \psi) (\partial_{ij} \psi)
    + \left( \frac12 \left(\nabla^2 \psi\right)^2 + \psi \nabla^2 \psi 
    + \frac12 (1 + \epsilon) \psi^2 + \frac14 \psi^4 \right) \delta_{ij}
\end{equation}
Finally, written in terms of our auxiliary variables:
\begin{equation}
    \sigma_{ij}
    =
    (\partial_i \psi) (\partial_j \psi)
    + (\partial_i \chi) (\partial_j \psi)
    - \psi \partial_{ij} \psi
    - \chi (\partial_{ij} \psi)
    + \tfrac12 \left(\chi^2 + 2 \psi \nabla^2 \psi + (1 + \epsilon) \psi^2 + \tfrac12 \psi^4 \right)\delta_{ij}
\end{equation}
Written in terms of general vector operators, we get the following:
\begin{equation}
    \sigma
    =
    \nabla \left(\psi + \chi \right) \otimes \nabla \psi
    - (\psi + \chi) \nabla(\nabla \psi)
    + \tfrac12 \left(\chi^2 + 2 \psi \nabla^2 \psi + (1 + \epsilon) \psi^2 + \tfrac12 \psi^4 \right) I
\end{equation}
Now note that we will not actually be able to calculate this at each of the quadrature points because we are only using linear elements.

\subsection{Weak form of the configurational stress}
Because we are using piecewise linear elements, we need to write down a weak form if we are to accurately calculate the higher order derivatives.
Taking the inner product with a tensor test function $\tau$ yields:
\begin{equation}
    \begin{split}
        \left< \tau, \sigma \right>
        &=
        \left< \tau, \nabla \left( \psi + \chi \right) \otimes (\nabla \psi) \right>
        - \left< \tau, (\psi + \chi) \nabla(\nabla \psi) \right>
        + \tfrac12 \left< \tau, \left(\chi^2 + 2 \psi \nabla^2 \psi + (1 + \epsilon) \psi^2 + \tfrac12 \psi^4 \right) I \right> \\
        &=
        \begin{multlined}[t]
            \left< \tau, \nabla \left( \psi + \chi \right) \otimes (\nabla \psi) \right>
            + \left< \nabla \cdot \left((\psi + \chi) \tau \right), \nabla \psi \right>
            - \left< (\psi + \chi) \mathbf{n} \cdot \tau, \nabla \psi \right>_{\partial \Omega} \\
            + \tfrac12 \left< \text{tr}(\tau), \chi^2 + 2 \psi \nabla^2 \psi + (1 + \epsilon) \psi^2 + \tfrac12 \psi^4  \right>
        \end{multlined} \\
        &=
        \begin{multlined}[t]
            \left< \tau, \nabla \left( \psi + \chi \right) \otimes (\nabla \psi) \right>
            + \left< (\nabla \psi + \nabla \chi) \cdot  \tau , \nabla \psi \right>
            + \left< (\psi + \chi) \nabla \cdot \tau, \nabla \psi \right>
            - \left< (\psi + \chi) \mathbf{n} \cdot \tau, \nabla \psi \right>_{\partial \Omega} \\
            + \tfrac12 \left< \text{tr}(\tau), \chi^2 + 2 \psi \nabla^2 \psi + (1 + \epsilon) \psi^2 + \tfrac12 \psi^4 \right>
        \end{multlined} \\
    \end{split}
\end{equation}
Given that we're working with periodic boundary conditions, there is no boundary so the equation reduces to:
\begin{equation}
    \left< \tau, \sigma \right>
    =
    \begin{multlined}[t]
        \left< \tau, \nabla \left( \psi + \chi \right) \otimes (\nabla \psi) \right>
        + \left< (\nabla \psi + \nabla \chi) \cdot  \tau , \nabla \psi \right>
        + \left< (\psi + \chi) \nabla \cdot \tau, \nabla \psi \right>\\
        + \tfrac12 \left< \text{tr}(\tau), \chi^2 + 2 \psi \nabla^2 \psi + (1 + \epsilon) \psi^2 + \tfrac12 \psi^4 \right>
    \end{multlined}
\end{equation}
To actually project this onto a finite element space, we consider an approximation of $\sigma$:
\begin{equation}
    \sigma_h = \sum_i \sigma_i \tau_i
\end{equation}
where each $\sigma_i$ is a scalar.
Then the problem becomes:
\begin{equation}
    M_{ij} \sigma_j = b_i
\end{equation}
where
\begin{equation}
    M_{ij}
    =
    \left<\tau_i, \tau_j\right>
\end{equation}
is the mass matrix and
\begin{equation}
    b_i
    =
    \begin{multlined}[t]
        \left< \tau_i, \nabla \left( \psi + \chi \right) \otimes (\nabla \psi) \right>
        + \left< (\nabla \psi + \nabla \chi) \cdot  \tau_i, \nabla \psi \right>
        + \left< (\psi + \chi) \nabla \cdot \tau_i, \nabla \psi \right>\\
        + \tfrac12 \left< \text{tr}(\tau_i), \chi^2 + 2 \psi \nabla^2 \psi + (1 + \epsilon) \psi^2 + \tfrac12 \psi^4 \right>
    \end{multlined}
\end{equation}

\section{Deriving the time evolution}
\subsection{Forcing free energy to be non-positive}
In general, we would like to impose that the free energy only decrease over time due to dissipative effects:
\begin{equation}
    \frac{\partial \mathcal{F}}{\partial t} \leq 0
\end{equation}
Additionally, we would like to impose conservation of momentum, which takes the form:
\begin{equation}
    \frac{\partial \psi}{\partial t} + \nabla \cdot \mathbf{J} = 0
\end{equation}
where $\mathbf{J}$ is the mass current.
One way to do this is very straightforward:
\begin{equation}
    \begin{split}
        \frac{\partial \mathcal{F}}{\partial t}
        &=
        \int_\Omega \frac{\delta \mathcal{F}}{\delta \psi} \frac{\partial \psi}{\partial t} dV \\
        &=
        -\int_\Omega \frac{\delta \mathcal{F}}{\delta \psi} \nabla \cdot \mathbf{J} \\
        &=
        \int_\Omega \nabla \frac{\delta \mathcal{F}}{\delta \psi} \cdot \mathbf{J}
    \end{split}
\end{equation}
where in the last step we have integrated by parts and assumed that the current is zero at the boundary.
To force this to be non-positive, we can take:
\begin{equation}
    \mathbf{J} = \nabla \frac{\delta \mathcal{F}}{\delta \psi}
    \implies
    \frac{\partial \psi}{\partial t} = \nabla^2 \frac{\delta \mathcal{F}}{\delta \psi}
\end{equation}

\subsection{Using Rayleigh dissipation function}
The above method is straightforward, but it does not generalize well to imposing additional constraints.
To this end, we rederive the equation of motion using the method of the Rayleigh dissipation function.
Following \href{https://iopscience.iop.org/article/10.1088/0953-8984/23/28/284118}{Doi}, we define the energy dissipation in terms of the phase-field current:
\begin{equation}
    W = \int_\Omega \xi J^2 dV
\end{equation}
with some frictional coefficient $\xi$, and $\mathbf{J}$ the phase-field current.
The free energy functional is given by:
\begin{equation}
    \mathcal{F}[\psi]
    =
    \int_\Omega f(\psi) dV
\end{equation}
From above, we have that:
\begin{equation}
    \frac{d \mathcal{F}}{dt}
    =
    \int_\Omega \left(\mathbf{J} \cdot \nabla \frac{\delta \mathcal{F}}{\delta \psi} \right)dV
\end{equation}
The Rayleighian is then given by:
\begin{equation}
    \mathcal{R}
    =
    \frac12 \int_\Omega \xi J^2 dV
    + \int_\Omega \left(\mathbf{J} \cdot \nabla \frac{\delta \mathcal{F}}{\delta \psi} \right)dV
\end{equation}
We can then minimize this by taking the derivative with respect to the current:
\begin{equation}
    \frac{\delta \mathcal{R}}{\delta \mathbf{J}}
    =
    \xi \mathbf{J}
    + \nabla \frac{\delta \mathcal{F}}{\delta \psi}
    =
    0   
\end{equation}
which implies:
\begin{equation}
    \mathbf{J}
    =
    -\frac{1}{\xi}
    \nabla \frac{\delta \mathcal{F}}{\delta \psi}
\end{equation}
Plugging this into the conservation equation gives us back the desired result.

\subsection{Adding additional constraints}
Given that we are able to use the Rayleigh dissipation function formalism to derive the equation of motion, we may add additional constraints via the method of Lagrange multipliers.
Here, the additional constraint is one on the configurational stress:
\begin{equation}
    \nabla \cdot \sigma = 0
\end{equation}
where $\sigma$ is as defined in \eqref{eq:configurational-stress}.
Then we write down a Lagrangian as:
\begin{equation}
    \mathcal{L}
    =
    \frac12 \int_\Omega \xi J^2 dV
    + 
    \int_\Omega \left( \mathbf{J} \cdot \nabla \frac{\delta \mathcal{F}}{\delta \psi} \right) dv
    -
    \int_\Omega \lambda(x) \nabla \cdot \sigma dV
\end{equation}

\section{Time evolution of this configuration}
For debugging purposes, let's calculate the right-hand side of the time evolution equation given this configuration:
\begin{equation*}
    \begin{split}
        \nabla^2 \phi
        &=
        -A_0 \sum_n q_n^6 \cos(\mathbf{q}_n \cdot \mathbf{r}) \\
        2 \phi
        &=
        2 A_0 \sum_n q_n^4 \cos(\mathbf{q}_n \cdot \mathbf{r}) \\
        (1 + \epsilon) \chi
        &=
        -(1 + \epsilon) A_0 \sum_n q_n^2 \cos(\mathbf{q}_n \cdot \mathbf{r}) \\
        %
        3 \psi^2 \chi
        &= 3\left(\psi_0 + A_0 \sum_l \cos(\mathbf{q}_l \cdot \mathbf{r})\right)
        \left(\psi_0 + A_0 \sum_m \cos(\mathbf{q}_m \cdot \mathbf{r})\right) 
        \left(-A_0 \sum_n q_n^2 \cos(\mathbf{q}_n \cdot \mathbf{r})\right) \\
        &= 3 \left(\psi_0^2 + 2 \psi_0 A_0 \sum_l \cos(\mathbf{q}_l \cdot \mathbf{r}) 
        + A_0^2 \sum_{l, m} \cos(\mathbf{q}_l \cdot \mathbf{r}) \cos(\mathbf{q}_m \cdot \mathbf{r})\right)
        \left(-A_0 \sum_n q_n^2 \cos(\mathbf{q}_n \cdot \mathbf{r})\right) \\
        &= -3 \left(\psi_0 A_0 \sum_n q_n^2 \cos(\mathbf{q}_n \cdot \mathbf{r})
        + 2 \psi_0 A_0^2 \sum_{n, l} \cos(\mathbf{q}_l \cdot \mathbf{r}) \cos(\mathbf{q}_n \cdot \mathbf{r})
        + A_0^3 \sum_{l, m, n} \cos(\mathbf{q}_l \cdot \mathbf{r})\cos(\mathbf{q}_m \cdot \mathbf{r})\cos(\mathbf{q}_n \cdot \mathbf{r}) \right)\\
        &=
        \begin{multlined}[t]
            -3 \biggl( \psi_0 A_0 \sum_n q_n^2 \cos(\mathbf{q}_n \cdot \mathbf{r}) \\
            + \psi_0 A_0^2 \sum_{n, l} \left[\cos((\mathbf{q}_n - \mathbf{q}_l) \cdot \mathbf{r}) + \cos((\mathbf{q}_n + \mathbf{q}_l) \cdot \mathbf{r})\right] \\
            + \frac12 A_0^3 \sum_{l, m, n} \cos(\mathbf{q}_m \cdot \mathbf{r}) \left[\cos((\mathbf{q}_n - \mathbf{q}_l) \cdot \mathbf{r}) + \cos((\mathbf{q}_n + \mathbf{q}_l) \cdot \mathbf{r})\right]
            \biggr)
        \end{multlined} \\
        %
        6 \psi \left(\nabla \psi \right)^2
        &=
        6 \left(\psi_0 + A_0 \sum_l \cos(\mathbf{q}_l \cdot \mathbf{r})\right)
        \left( A_0^2 \sum_{n, m} \mathbf{q}_n \cdot \mathbf{q}_m \sin(\mathbf{q}_n \cdot \mathbf{r}) \sin(\mathbf{q}_m \cdot \mathbf{r}) \right) \\
        &= 
        \begin{multlined}[t]
        3 \biggl(
        \psi_0 A_0^2 \sum_{n, m} \left(\frac32 \delta_{m, n} - \frac12\right) \left[\cos((\mathbf{q}_m - \mathbf{q}_n)\cdot \mathbf{r}) - \cos((\mathbf{q}_m + \mathbf{q}_n)\cdot \mathbf{r}) \right] \\
        + A_0^3 \sum_{l, m, n} \left(\frac32 \delta_{m, n} - \frac12\right) \cos(\mathbf{q}_l \cdot \mathbf{r}) \left[\cos((\mathbf{q}_m - \mathbf{q}_n)\cdot \mathbf{r}) - \cos((\mathbf{q}_m + \mathbf{q}_n)\cdot \mathbf{r}) \right)
        \biggr)
        \end{multlined}
    \end{split}
\end{equation*}
First we note that, because $q_n^2 = 1$ we get:
\begin{equation*}
    \nabla^2 \phi + 2 \phi + (1 + \epsilon) \chi
    =
    \epsilon A_0 \sum_n \cos(\mathbf{q}_n \cdot \mathbf{r})
\end{equation*}
Then, taking $\psi_0 = 0$ for ease of calculation we get:
\begin{equation}
    3 \psi^2 \chi + 6 \psi \left(\nabla \psi\right)^2
    =
    \frac92 A_0^3 \sum_l \cos(\mathbf{q}_l \cdot \mathbf{r})
\end{equation}

For debugging purposes, let's calculate the right-hand side of the time evolution equation given this configuration.
For ease of calculation, we take the complex exponential with the understanding that at the end we take the real part of the entire expression:
\begin{equation*}
    \nabla^2 \phi
    = - A_0 \sum_n q_n^6 e^{i \mathbf{q}_n \cdot \mathbf{r}}
\end{equation*}
\begin{equation*}
    2 \phi
    = 2 A_0 \sum_n q_n^4 e^{i \mathbf{q}_n \cdot \mathbf{r}}
\end{equation*}
\begin{equation*}
    (1 + \epsilon) \chi
    = -(1 + \epsilon) A_0 \sum_n q_n^2 e^{i \mathbf{q}_n \cdot \mathbf{r}}
\end{equation*}
Thus:
\begin{equation}
    \nabla^2 \phi + 2\phi +(1 + \epsilon)\chi
    = -\epsilon A_0 \sum_n q_n^2 e^{i \mathbf{q}_n \cdot \mathbf{r}}
\end{equation}
Now for the nonlinear terms:
\begin{equation*}
    \begin{split}
        3 \psi^2 \chi
        &=
        3 \left(\psi_0 + A_0 \sum_n e^{i \mathbf{q}_n \cdot \mathbf{r}}\right)
        \left(\psi_0 + A_0 \sum_m e^{i \mathbf{q}_m \cdot \mathbf{r}}\right)
        \left(-A_0 \sum_l e^{i \mathbf{q}_l \cdot \mathbf{r}} \right) \\
        &=
        3\left(\psi_0^2 
        + 2 \psi_0 A_0 \sum_n e^{i \mathbf{q}_n \cdot \mathbf{r}}
        + A_0^2 \sum_{n, m} e^{i \left(\mathbf{q}_n + \mathbf{q}_m\right) \cdot \mathbf{r}}
        \right)
        \left(-A_0 \sum_l e^{i \mathbf{q}_l \cdot \mathbf{r}} \right) \\
        &=
        -3 \psi_0^2 A_0 \sum_l e^{\mathbf{q}_l \cdot \mathbf{r}}
        - 6 \psi_0 A_0^2 \sum_{n, m} e^{i \left(\mathbf{q}_n + \mathbf{q}_m\right) \cdot \mathbf{r}}
        -3 A_0^3 \sum_{n, m, l} e^{i \left(\mathbf{q}_n + \mathbf{q}_m + \mathbf{q}_l\right) \cdot \mathbf{r}}
    \end{split}
\end{equation*}
\begin{equation*}
    \begin{split}
        6 \psi \left(\nabla \psi\right)^2
        &=
        6 \left(\psi_0 + A_0 \sum_n e^{i \mathbf{q}_n \cdot \mathbf{r}} \right)
        \left(A_0 \sum_m i\mathbf{q}_m e^{i \mathbf{q}_m \cdot \mathbf{r}} \right) \cdot
        \left(A_0 \sum_l i\mathbf{q}_l e^{i \mathbf{q}_l \cdot \mathbf{r}} \right) \\
        &=
        -6\psi_0 A_0^2 \sum_{n, m} \mathbf{q}_n \cdot \mathbf{q}_m e^{i \left(\mathbf{q}_n + \mathbf{q}_m\right) \cdot \mathbf{r}}
        -6 A_0^3 \sum_{n, m, l} \mathbf{q}_n \cdot \mathbf{q}_m e^{i \left(\mathbf{q}_n + \mathbf{q}_m + \mathbf{q}_l\right) \cdot \mathbf{r}}
    \end{split}
\end{equation*}
Now we have to establish some identities of the lattice vectors for this configuration.
Note that:
\begin{equation}
    \begin{split}
    \mathbf{q}_m \cdot \mathbf{q}_n
        &=
    \begin{cases}
        1 & m = n \\
        -\frac12 & m \neq n
    \end{cases} \\
        &=
    \left(\frac32 \delta_{mn} - \frac12\right)
    \end{split}
\end{equation}
Additionally,
\begin{equation}
    \mathbf{q}_m + \mathbf{q}_n
    =
    \begin{cases}
        2 \mathbf{q}_m & m = n \\
        -\mathbf{q}_l & m \neq n \text{, (where } m \neq l \neq n \text{)}
    \end{cases}
\end{equation}
Given these, we may make some progress:
\begin{equation}
    \begin{split}
        \sum_{n, m} \mathbf{q}_n \cdot \mathbf{q}_n e^{i \left(\mathbf{q}_n + \mathbf{q}_m\right) \cdot \mathbf{r}}
        &=
        \sum_n e^{2 \mathbf{q}_n \cdot \mathbf{r}}
        - \frac12 \sum_{m \neq n} e^{i \left(\mathbf{q}_n + \mathbf{q}_m \right) \cdot \mathbf{r}} \\
        &=
        \sum_n \left(e^{2 i \mathbf{q}_n \cdot \mathbf{r}} - e^{- i \mathbf{q_n \cdot \mathbf{r}}}\right)
    \end{split}
\end{equation}
Additionally,
\begin{equation}
    \begin{split}
        \sum_{n, m, l} \mathbf{q}_n \cdot \mathbf{q}_m e^{i \left( \mathbf{q}_n + \mathbf{q}_m + \mathbf{q}_l \right) \cdot \mathbf{r} }
        &=
        \sum_{n, m} \left(e^{2 i \mathbf{q}_n \cdot \mathbf{r}} - e^{-i \mathbf{q}_n \cdot r} \right)
        e^{i \mathbf{q}_m \cdot \mathbf{r}} \\
        &=
        \sum_{n, m} e^{i \left(2 \mathbf{q}_n + \mathbf{q}_m \right) \cdot \mathbf{r}}
        - \sum_{n, m} e^{i \left(\mathbf{q}_n - \mathbf{q}_m \right) \cdot \mathbf{r}}
    \end{split}
\end{equation}
Also:
\begin{equation}
    \begin{split}
        \sum_{n, m} e^{i \left(\mathbf{q}_n + \mathbf{q}_m\right) \cdot \mathbf{r}}
        &=
        \sum_n e^{i 2 \mathbf{q}_n \cdot \mathbf{r}}
        + 2 \sum_n e^{-i \mathbf{q}_n \cdot \mathbf{r}}
    \end{split}
\end{equation}
and finally:
\begin{equation}
    \begin{split}
        \sum_{n, m, l} e^{i \left( \mathbf{q}_n + \mathbf{q}_m + \mathbf{q}_l\right) \cdot \mathbf{r}}
        &=
        \sum_{n, m} \left(e^{i 2 \mathbf{q}_n \cdot \mathbf{r}} + 2 e^{-i \mathbf{q}_n \cdot \mathbf{r}}\right)
        e^{i \mathbf{q}_m \cdot \mathbf{r}} \\
        &=
        \sum_{n, m} e^{i \left(2 \mathbf{q}_n + \mathbf{q}_m \right) \cdot \mathbf{r}}
        + 2 \sum_{n, m} e^{i \left(\mathbf{q}_n - \mathbf{q}_m \right) \cdot \mathbf{r}}
    \end{split}
\end{equation}
Substituting we get:
\begin{equation}
    6 \psi \left(\nabla \psi\right)^2
    =
    -6 \psi_0 A_0^2 \sum_n e^{2 i \mathbf{q}_n \cdot \mathbf{r}}
    + 6 \psi_0 A_0^2 \sum_n e^{- i \mathbf{q_n \cdot \mathbf{r}}}
    - 6 A_0^3 \sum_{n, m} e^{i \left(2 \mathbf{q}_n + \mathbf{q}_m \right) \cdot \mathbf{r}}
    + 6 A_0^3 \sum_{n, m} e^{i \left(\mathbf{q}_n - \mathbf{q}_m \right) \cdot \mathbf{r}}
\end{equation}
and
\begin{equation}
    3 \psi^2 \chi
    =
    -3 \psi_0^2 A_0 \sum_n e^{i \mathbf{q}_n \cdot \mathbf{r}}
    -6 \psi_0 A_0^2 \sum_n e^{i 2 \mathbf{q}_n \cdot \mathbf{r}}
    - 12\psi_0 A_0^2 \sum_n e^{-i \mathbf{q}_n \cdot \mathbf{r}}
    - 3 A_0^3 \sum_{n, m} e^{i \left(2 \mathbf{q}_n + \mathbf{q}_m \right) \cdot \mathbf{r}}
    - 6 A_0^3 \sum_{n, m} e^{i \left(\mathbf{q}_n - \mathbf{q}_m \right) \cdot \mathbf{r}}
\end{equation}

\end{document}
