\documentclass[reqno]{article}
\usepackage{../format_doc}

\begin{document}
\title{Finite element local convolution integrals}
\author{Lucas Myers}
\maketitle

\section{Convolution definition}

In these kinds of problems where there's a long-wavelength distortion in an otherwise periodic field, we have the need to take an average over the system period.
To do this in a smooth way, one may define the local averaging as:
\begin{equation} \label{eq:convolution-integral}
    \tilde{X} (\mathbf{r})
    =
    \int_\Omega d\mathbf{r}' \, 
    \frac{X(\mathbf{r}')}{\left( 2 \pi a_0^2 \right)^{d/2}}
    \exp\left( 
        - \frac{ \left( \mathbf{r} - \mathbf{r}' \right)^2 }{2 a_0^2}
    \right)
\end{equation}
where $\Omega$ is the domain, and $a_0$ is the lattice spacing.
Typically the averaging is denoted by $\left< X \right>$, but we have use for the angle brackets when taking the inner product of functions.
Note that, since the integrand decays exponentially away from $\mathbf{r}$, we may define some cutoff $\Lambda$ at which distance away from $\mathbf{r}$ we truncate the integral.
Then this integral is done over a domain $C(\Lambda, \mathbf{r})$, a circle of radius $\Lambda$ centered at $\mathbf{r}$.

Approximating the local average in terms of Lagrange elements $\phi$ yields:
\begin{equation}
    \tilde{X}
    \approx
    \sum_j \tilde{X}_j \phi_j
\end{equation}
Taking the inner product with other Lagrange elements $\phi_i$ yields:
\begin{equation}
    \sum_j \tilde{X}_j \left< \phi_i, \phi_j\right>
    =
    \left< \phi_i, \tilde{X}\right>
\end{equation}
where we are now taking this as the definition of $\tilde{X}_j$.
We may easily solve this since $M_{ij} = \left< \phi_i, \phi_j \right>$ is symmetric and positive-definite.
However, to implement this numerically we will need to query $\tilde{X}$ at every quadrature point, which corresponds to calculating \eqref{eq:convolution-integral} at every quadrature point.
This is actually tractable by introducing a cutoff, but we still have to keep track of $\tilde{X}$ at every quadrature point.

Explicitly the numerical computation of the right-hand side is given by:
\begin{equation}
    \begin{split}
        \left< \phi_i, \tilde{X}\right>
        &=
        \sum_q \phi_i^{(q)} \tilde{X}^{(q)} \left( J \times W \right)^{(q)} \\
        &=
        \sum_q \phi_i^{(q)} 
            \sum_{q'} \frac{1}{\left( 2 \pi a_0^2 \right)} X^{(q')} \exp \left( - \frac{(\mathbf{r}^{(q)} - \mathbf{r}^{(q')})^2}{2 a_0^2} \right) 
            \left( J \times W \right)^{(q')} 
        \left( J \times W \right)^{(q)}\\
        &= 
        \frac{1}{\left( 2 \pi a_0^2 \right)} 
        \sum_{q'} \sum_q \phi_i^{(q)}
            X^{(q')} \exp \left( - \frac{(\mathbf{r}^{(q)} - \mathbf{r}^{(q')})^2}{2 a_0^2} \right) 
            \left( J \times W \right)^{(q')}
        \left( J \times W \right)^{(q)}
        \end{split}
\end{equation}
where here $q$ and $q'$ are indices of the two sets of quadrature points, a superscript of each of them corresponds to indexing by them, and $(J \times W)$ is the corresponding quadrature weight (scaled by the Jacobian from the unit cell).
So then we must do one of two things in the multi-MPI process implementation: either we must send $\mathbf{r}^{(q)}$ to each process, calculate $\sum_q' X^{(q')} \exp\left( -(\mathbf{r}^{(q)} - \mathbf{r}^{(q')})^2 / 2 a_0^2 \right) (J \times W)^{(q')}$ on the other processes, then reassociate them with each quadrature point $q$ so that the $q$ summation can be carried out; or we may send over a set of $\phi_i^{(q)}$, $\mathbf{r}^{(q)}$, and $(J \times W)^{(q)}$, evaluate both sums on the external process, and then just reassociate with the local DoF $i$.
I think it is probably easiest to do the former, and also it avoids having to send \verb|n_dofs| $\times$ \verb|n_quad_points| test functions.

\end{document}
